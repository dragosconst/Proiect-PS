\documentclass[12pt]{article}
\usepackage{fancyhdr}
\usepackage{amsmath}
\usepackage{amsthm}
\usepackage{mathtools}
\usepackage{enumitem}
\usepackage[Export]{adjustbox}
\usepackage{cancel}
\usepackage{algorithm}
\usepackage[noend]{algpseudocode}
\usepackage{graphicx}
\usepackage[margin = 1in]{geometry}
\usepackage{blindtext}
\usepackage[section]{placeins}
\usepackage{xcolor,listings}
\usepackage{textcomp}
\usepackage[utf8]{inputenc}
\lstset{upquote=true}
\lstdefinestyle{myCustomRStyle}{
	language=R,
	backgroundcolor = \color{lightgray!10!white},
	numbers=left,
	stepnumber=1,
	numbersep=10pt,
	tabsize=2,
	showspaces=false,
	breaklines=true
	showstringspaces=false,
	basicstyle=\footnotesize\ttfamily,
	keywordstyle=\bfseries\color{blue!50!black},
	commentstyle=\itshape\color{orange!60!black},
	identifierstyle=\color{black},
	stringstyle=\color{green!50!black},
	xleftmargin=\parindent,
	frame=L
}
\lstset{literate=%
	*{0}{{{\color{blue}0}}}1
	{1}{{{\color{blue}1}}}1
	{2}{{{\color{blue}2}}}1
	{3}{{{\color{blue}3}}}1
	{4}{{{\color{blue}4}}}1
	{5}{{{\color{blue}5}}}1
	{6}{{{\color{blue}6}}}1
	{7}{{{\color{blue}7}}}1
	{8}{{{\color{blue}8}}}1
	{9}{{{\color{blue}9}}}1
}

\lstset{basicstyle=\small,style=myCustomRStyle}
\usepackage{amsfonts}
\graphicspath{ {./images/} }

\pagestyle{fancy}
\fancyhead{}
\fancyfoot{}
\usepackage[T1]{fontenc}
\fancyhead[L]{Placeholder}
\fancyhead[R]{Placeholder}
\setlength{\headheight}{25pt}
\fancyfoot[C]{\thepage}
\title{Placeholder}
\author{Placeholder}

\begin{document}
	\textbf{Calculul covarianței și coeficientului de corelație pentru două variabile aleatoare
		continue(Atenție: Trebuie să folosiți densitatea comună a celor două variabile
		aleatoare!)
}\vspace{5mm}
	
	\section*{Covarianță}
	\begin{center}
		\begin{tabular}{|| c | c | c ||}
			\hline
			Parametrul & Tipul & Descriere \\
			\hline
			\textcolor{blue}{Z} & \textcolor{violet}{contRV} & Variabila aleatoare continuă bidimensională $Z = (X, Y)$ \\
			
			\hline
		\end{tabular}
	\end{center}\hfill \\

	Funcția \texttt{Cov} calculează covarianța pe baza formulei:
	\begin{equation*}
		Cov(X, Y) = E[XY] - E[X] \cdot E[Y]
	\end{equation*}

	\begin{lstlisting}
		Cov <- function(Z) {
			
			if (!Z@bidimen) {
				print("Variabila nu este bidimensionala!")
				NA
			}
			else {
				
				X <- marginalaX(Z)
				Y <- marginalaY(Z)
				
				return (E(Z) - E(X) * E(Y))
				
			}
		}
	\end{lstlisting}

	\section*{Coeficientul de corelație}
	\begin{center}
		\begin{tabular}{|| c | c | c ||}
			\hline
			Parametrul & Tipul & Descriere \\
			\hline
			\textcolor{blue}{Z} & \textcolor{violet}{contRV} & Variabila aleatoare continuă bidimensională $Z = (X, Y)$ \\
			
			\hline
		\end{tabular}
	\end{center}\hfill \\
	
	Funcția \texttt{Cor} calculează coeficientul de corelație pe baza formulei:
	\begin{equation*}
		\rho(X, Y) = \frac{Cov(X, Y)}{\sqrt{Var(X) \cdot Var(Y))}}
	\end{equation*}\pagebreak
	
	\begin{lstlisting}
		Cor <- function(Z)
		{
			if (!Z@bidimen) {
				print("Variabila nu este bidimensionala!")
				NA
			}
			else {
				
				X <- marginalaX(Z)
				Y <- marginalaY(Z)
				
				return (Cov(Z) / sqrt(Var(X) * Var(Y)))
				
			}
		}
	\end{lstlisting}\vspace*{2\baselineskip}
	
	Exemple:
	\begin{lstlisting}
		> Z <- contRV(densitate = function (x, y) (6/7) * (x+y)^2,
		             bidimen = TRUE,
		             suport = list(list(c(0, 1)), list(c(0, 1))))
		> Cov(Z)
		[1] -0.008503401
		> Cor(Z)
		[1] -0.1256281
	\end{lstlisting}
	
\end{document}