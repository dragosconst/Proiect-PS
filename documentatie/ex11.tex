\documentclass[12pt]{article}
\usepackage{fancyhdr}
\usepackage{amsmath}
\usepackage{amsthm}
\usepackage{mathtools}
\usepackage{enumitem}
\usepackage[Export]{adjustbox}
\usepackage{cancel}
\usepackage{algorithm}
\usepackage{bigints}
\usepackage[noend]{algpseudocode}
\usepackage{graphicx}
\usepackage[margin = 1in]{geometry}
\usepackage{blindtext}
\usepackage[section]{placeins}
\usepackage{xcolor,listings}
\usepackage{hyperref}
\usepackage{textcomp}
\usepackage{makecell}
\usepackage[utf8]{inputenc}
\lstset{upquote=true}
\lstdefinestyle{myCustomRStyle}{
	language=R,
	backgroundcolor = \color{lightgray!10!white},
	numbers=left,
	stepnumber=1,
	numbersep=10pt,
	tabsize=2,
	showspaces=false,
	breaklines=true
	showstringspaces=false,
	basicstyle=\footnotesize\ttfamily,
	keywordstyle=\bfseries\color{blue!50!black},
	commentstyle=\itshape\color{orange!60!black},
	identifierstyle=\color{black},
	stringstyle=\color{green!50!black},
	xleftmargin=\parindent,
	frame=L
}
\lstset{literate=%
	*{0}{{{\color{blue}0}}}1
	{1}{{{\color{blue}1}}}1
	{2}{{{\color{blue}2}}}1
	{3}{{{\color{blue}3}}}1
	{4}{{{\color{blue}4}}}1
	{5}{{{\color{blue}5}}}1
	{6}{{{\color{blue}6}}}1
	{7}{{{\color{blue}7}}}1
	{8}{{{\color{blue}8}}}1
	{9}{{{\color{blue}9}}}1
}

\lstset{basicstyle=\small,style=myCustomRStyle}
\usepackage{amsfonts}


\pagestyle{fancy}
\fancyhead{}
\fancyfoot{}
\usepackage[T1]{fontenc}
\fancyhead[L]{Placeholder}
\fancyhead[R]{Placeholder}
\setlength{\headheight}{25pt}
\fancyfoot[C]{\thepage}
\title{Bază de date pentru gestionarea unei afaceri cu librării}
\author{Placeholder}

\begin{document}
\section{Cerința 11}
\hfill \textbf{1)Pornind  de  la  densitatea  comună  a  două  variabile  aleatoare  continue,  construirea densităților marginale și a densităților condiționate.} \hfill \\
\indent Implementarea densitățiilor marginale și condiționate se fac în funcție de funcția $integrala$. Astfel, avem densitățiile marginale:
\begin{lstlisting}
# construieste o v.a continua pornind de la densitatea marginala a lui X in v.a bidimen (X, Y)
marginalaX <- function(XY)
{
	return (contRV(densitate = integrala(XY, 2), suport = XY@suport[[1]], bidimen = FALSE, ref_va_bidimen = XY))
}

# construieste o v.a continua pornind de la densitatea marginala a lui Y in v.a bidimen (X, Y)
marginalaY <- function(XY)
{
	return (contRV(densitate = integrala(XY, 1), suport = XY@suport[[2]], bidimen = FALSE, ref_va_bidimen = XY))
}
\end{lstlisting} \hfill \\
\indent Câmpul $ref\_va\_bidimen$ este folosit pentru a putea accesa cuplul $(X, Y)$ din $X$ sau $Y$ când vrem să calculăm vreo probabilitate care depinde de ambele. \\
\indent Densitățiile condiționate au fost implementate în mod asemănător:
\begin{lstlisting}
dens_condit_x_de_y <- function(Z)
{
	dens_marginala_y <- integrala(Z,1)
	f <- function(x,y) {if (dens_marginala_y(y) != 0) {Z@densitate(x,y) / dens_marginala_y(y)} else stop("Densitatea marginala este 0")}
	return (Vectorize(f))
}

dens_condit_y_de_x <- function(Z)
{
	dens_marginala_x <- integrala(Z,2)
	f <- function(x,y) {if (dens_marginala_y(y) != 0) {Z@densitate(x,y) / dens_marginala_x(x)} else stop("Densitatea marginala este 0")}
	return (Vectorize(f))
}
\end{lstlisting}
\end{document}