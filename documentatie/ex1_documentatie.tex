\documentclass[12pt]{article}
\usepackage{fancyhdr}
\usepackage{amsmath}
\usepackage{amsthm}
\usepackage{mathtools}
\usepackage{enumitem}
\usepackage[Export]{adjustbox}
\usepackage{cancel}
\usepackage{algorithm}
\usepackage[noend]{algpseudocode}
\usepackage{graphicx}
\usepackage[margin = 1in]{geometry}
\usepackage{blindtext}
\usepackage[section]{placeins}
\usepackage{xcolor,listings}
\usepackage{textcomp}
\usepackage[utf8]{inputenc}
\lstset{upquote=true}
\lstdefinestyle{myCustomRStyle}{
	language=R,
	backgroundcolor = \color{lightgray!10!white},
	numbers=left,
	stepnumber=1,
	numbersep=10pt,
	tabsize=2,
	showspaces=false,
	breaklines=true
	showstringspaces=false,
	basicstyle=\footnotesize\ttfamily,
	keywordstyle=\bfseries\color{blue!50!black},
	commentstyle=\itshape\color{orange!60!black},
	identifierstyle=\color{black},
	stringstyle=\color{green!50!black},
	xleftmargin=\parindent,
	frame=L
}
\lstset{literate=%
	*{0}{{{\color{blue}0}}}1
	{1}{{{\color{blue}1}}}1
	{2}{{{\color{blue}2}}}1
	{3}{{{\color{blue}3}}}1
	{4}{{{\color{blue}4}}}1
	{5}{{{\color{blue}5}}}1
	{6}{{{\color{blue}6}}}1
	{7}{{{\color{blue}7}}}1
	{8}{{{\color{blue}8}}}1
	{9}{{{\color{blue}9}}}1
}

\lstset{basicstyle=\small,style=myCustomRStyle}
\usepackage{amsfonts}
\graphicspath{ {./images/} }

\pagestyle{fancy}
\fancyhead{}
\fancyfoot{}
\usepackage[T1]{fontenc}
\fancyhead[L]{Placeholder}
\fancyhead[R]{Placeholder}
\setlength{\headheight}{25pt}
\fancyfoot[C]{\thepage}
\title{Placeholder}
\author{Placeholder}
\begin{document}
	
\textbf{1) Fiind dată o funcție f , introdusă de utilizator, determinarea unei constante de
	normalizare k. În cazul în care o asemenea constantă nu există, afișarea unui mesaj
	corespunzător către utilizator}\vspace{5mm}

\indent Scurtă descriere a funcției: \\
\begin{center}
	\begin{tabular}{|| c | c | c ||}
		\hline
		Parametrul & Tipul & Descriere \\
		\hline
		\textcolor{blue}{Func} & \textcolor{violet}{function} & Funcția data ca parametru \\
		\hline
		\textcolor{blue}{sup} & \textcolor{violet}{listă de liste} & Suportul funcției \\
		\hline
	\end{tabular}
\end{center}
\begin{lstlisting}
Nor_constant <- function(Func, sup) {
	
	sum <- 0
	
	for (i in sup) {
		tryCatch(
		sum <- sum + integrate(Vectorize(Func), i[1], i[2], abs.tol = 0)$value,
		error= function(err) {
			stop("Integrala e divergenta sau functia nu e integrabila") # daca integrala nu poate fi calculata returnez un mesaj de eroare
		}
		)
		
		if (i[1] == -Inf && i[2] == Inf) {
			i[1] <- -1000
			i[2] <-  1000
		}
		else if (i[1] == -Inf)
		i[1] <- i[2] - 1000
		else if (i[2] ==  Inf)
		i[2] <- i[1] + 1000
		
		if(any(sapply(seq(i[1], i[2], length.out = 1000), Func) < 0))
		stop("Functie negativa") # daca functia are valori negative nu pot calcula constanta de normalizare
		
	}
	
	if(sum == 0)
	stop("Nu exista constanta de normalizare pentru functia data") # daca integrala = 0 inseamna ca nu exista constanta de normalizare
	
	const <- 1 / sum
	return (const)
	
}
\end{lstlisting}

\hfill Va fi parcurs suportul funcției dată ca parametru, și vom calcula suma pe fiecare interval din suport. În același timp, verificăm o daca funcția e pozitivă pe fiecare interval din suport, folosind $any(sapply(seq(i[1], i[2], length.out = 1000), Func) < 0)$, pentru a alege 1000 de valori echidistante.\hfill \\
\indent La final, dacă integrala este 0 înseamnă că nu există constantă de normalizare și afișează un mesaj corespunzător. În caz contrar, calculeză constanta.\hfill \\
\end{document}