\documentclass[12pt]{article}
\usepackage{fancyhdr}
\usepackage{amsmath}
\usepackage{amsthm}
\usepackage{mathtools}
\usepackage{enumitem}
\usepackage[Export]{adjustbox}
\usepackage{cancel}
\usepackage{algorithm}
\usepackage[noend]{algpseudocode}
\usepackage{graphicx}
\usepackage[margin = 1in]{geometry}
\usepackage{blindtext}
\usepackage[section]{placeins}
\usepackage{xcolor,listings}
\usepackage{textcomp}
\usepackage[utf8]{inputenc}
\lstset{upquote=true}
\lstdefinestyle{myCustomRStyle}{
	language=R,
	backgroundcolor = \color{lightgray!10!white},
	numbers=left,
	stepnumber=1,
	numbersep=10pt,
	tabsize=2,
	showspaces=false,
	breaklines=true
	showstringspaces=false,
	basicstyle=\footnotesize\ttfamily,
	keywordstyle=\bfseries\color{blue!50!black},
	commentstyle=\itshape\color{orange!60!black},
	identifierstyle=\color{black},
	stringstyle=\color{green!50!black},
	xleftmargin=\parindent,
	frame=L
}
\lstset{literate=%
	*{0}{{{\color{blue}0}}}1
	{1}{{{\color{blue}1}}}1
	{2}{{{\color{blue}2}}}1
	{3}{{{\color{blue}3}}}1
	{4}{{{\color{blue}4}}}1
	{5}{{{\color{blue}5}}}1
	{6}{{{\color{blue}6}}}1
	{7}{{{\color{blue}7}}}1
	{8}{{{\color{blue}8}}}1
	{9}{{{\color{blue}9}}}1
}

\lstset{basicstyle=\small,style=myCustomRStyle}
\usepackage{amsfonts}


\pagestyle{fancy}
\fancyhead{}
\fancyfoot{}
\usepackage[T1]{fontenc}
\fancyhead[L]{Placeholder}
\fancyhead[R]{Placeholder}
\setlength{\headheight}{25pt}
\fancyfoot[C]{\thepage}
\title{Bază de date pentru gestionarea unei afaceri cu librării}
\author{Placeholder}

\begin{document}
	%\section*{Cerința 2}
	\textbf{Verificarea dacă o funcție introdusă de utilizator este densitate de probabilitate.}\vspace{5mm}
	
	Dificultate întâmpinată: funcția \lstinline|integrate| returnează valori eronate pentru integranzi cu valoarea $0$ într-o mare parte a domeniului. Ca remediu, am ales să transmitem printr-un parametru suportul funcției de integrat (detalii în comentariul din cod).\par
	De asemenea, în cazul funcțiilor pe ramuri, \lstinline|Vectorize()| devine o necesitate: evaluarea condițiilor din if-uri (de exemplu, $x > 0$) ia în considerare doar primul element al vectorului Boolean $x > 0$. Apare astfel conflict cu modul de lucru al procedurii „integrate”, care evaluează funcția-argument pe un vector de mostre, nu punct-cu-punct.\par
	Definiția funcției este următoarea: \\
	
	\begin{lstlisting}
		# Suportul functiei f este reuniunea intervalelor specificate prin parametrul
		# sup - o lista de vectori de cate doua elemente, reprezentand extremitati de
		# interval. f este densitate de probabilitate - deci dp(f, sup)==TRUE - daca:
		# a) f(x) >= 0 pentru orice x din suport,
		# b) suma integralelor de f peste fiecare interval din sup este 1.
		
		# Din documentatia pentru integrate: "f must accept a vector of inputs and
		# produce a vector of function evaluations at those points". Considerand ca
		# utilizatorul introduce functia dorita in regim scalar -> scalar, aplicand
		# Vectorize() se obtine argumentul dorit pentru integrate.
		dp <- function(f, sup) {
			
			sum <- 0
			for (i in sup) {
				tryCatch(
				sum <- sum + integrate(Vectorize(f), i[1], i[2], abs.tol = 0)$value == 1,
				error = function(e) {
					print(sprintf("Integrala divergenta in intervalul [%.2f, %.2f]!", i[1], i[2]))
					return(FALSE)
				})
				
				if (i[1] == -Inf && i[2] == Inf) {
					i[1] <- -1000
					i[2] <-  1000
				}
				else if (i[1] == -Inf)
					i[1] <- i[2] - 1000
				else if (i[2] ==  Inf)
					i[2] <- i[1] + 1000
				
				if (any(sapply(seq(i[1], i[2], length.out = 1000), f) < 0)) {
					print(sprintf("Valoare negativa in intervalul [%.2f, %.2f]!", i[1], i[2]))
					return(FALSE)
				}
			}
			sum == 1
		}
	\end{lstlisting}\vspace*{3\baselineskip} 

	Exemple:
	\begin{lstlisting}[numbers=none]
		> dp(function(x) 3*x^2, list(c(0, 1)))
		[1] TRUE
		
		> f <- function(x) if (x < 0) 1 + x else 1 - x
		> dp(f, list(c(-1, 1)))
		[1] TRUE
		
		> dp(function(x) x, list(c(1, 3), c(-1, 0)))
		[1] "Valoare negativa in intervalul [-1.00, 0.00]!"
		[1] FALSE
		
		> dp(function(x) x, list(c(0, Inf)))
		[1] "Integrala divergenta in intervalul [0.00, Inf]!"
		[1] FALSE
	\end{lstlisting}
	
\end{document}