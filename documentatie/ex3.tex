\documentclass[12pt]{article}
\usepackage{fancyhdr}
\usepackage{amsmath}
\usepackage{amsthm}
\usepackage{mathtools}
\usepackage{enumitem}
\usepackage[Export]{adjustbox}
\usepackage{cancel}
\usepackage{algorithm}
\usepackage[noend]{algpseudocode}
\usepackage{graphicx}
\usepackage[margin = 1in]{geometry}
\usepackage{blindtext}
\usepackage[section]{placeins}
\usepackage{xcolor,listings}
\usepackage{textcomp}
\usepackage[utf8]{inputenc}
\lstset{upquote=true}
\lstdefinestyle{myCustomRStyle}{
	language=R,
	backgroundcolor = \color{lightgray!10!white},
	numbers=left,
	stepnumber=1,
	numbersep=10pt,
	tabsize=2,
	showspaces=false,
	breaklines=true
	showstringspaces=false,
	basicstyle=\footnotesize\ttfamily,
	keywordstyle=\bfseries\color{blue!50!black},
	commentstyle=\itshape\color{orange!60!black},
	identifierstyle=\color{black},
	stringstyle=\color{green!50!black},
	xleftmargin=\parindent,
	frame=L
}
\lstset{literate=%
	*{0}{{{\color{blue}0}}}1
	{1}{{{\color{blue}1}}}1
	{2}{{{\color{blue}2}}}1
	{3}{{{\color{blue}3}}}1
	{4}{{{\color{blue}4}}}1
	{5}{{{\color{blue}5}}}1
	{6}{{{\color{blue}6}}}1
	{7}{{{\color{blue}7}}}1
	{8}{{{\color{blue}8}}}1
	{9}{{{\color{blue}9}}}1
}

\lstset{basicstyle=\small,style=myCustomRStyle}
\usepackage{amsfonts}
\graphicspath{ {./images/} }

\pagestyle{fancy}
\fancyhead{}
\fancyfoot{}
\usepackage[T1]{fontenc}
\fancyhead[L]{Placeholder}
\fancyhead[R]{Placeholder}
\setlength{\headheight}{25pt}
\fancyfoot[C]{\thepage}
\title{Placeholder}
\author{Placeholder}

\begin{document}
	%\section*{Cerința 4}
	\textbf{Crearea unui obiect de tip variabilă aleatoare continuă pornind de la o densitate de
		probabilitate introdusă de utilizator. Funcția trebuie să aibă opțiunea pentru variabile
		aleatoare unidimensionale și respectiv bidimensionale.}\vspace{5mm}
	
	Pentru a facilita lucrul cu variabile aleatoare continue atât unidimensionale, cât si bidimensionale, am definit clasa de tip S4 numită \textbf{contRV}. Toate instanțele acestei clase vor reține o mulțime de informații necesare pentru a efectua diverse operații pe variabile aleatoare continue. Mai precis, fiecare obiect \textbf{contRV} va avea 
	asociat:
	\begin{itemize}
		\item O densitate de probabilitate.
		\item Un Boolean ce indică dacă variabila aleatoare este bidimensională.
		\item Suportul densității de probabilitate. În cazul variabilelor unidimensionale, acesta este reprezentat de o listă de intervale închise. Pentru o v.a bidimensională $(X, Y)$, suportul este reținut sub forma unei liste ce conține suporturile lui $X$ și $Y$.
		\item Pentru o v.a unidimensională $X$, o referință către v.a bidimensională $(X, Y)$, în cazul in care $X$ s-a format în urma determinării densității marginale. Se folosește pentru a avea acces cu ușurință la densitatea comună în calculul unor probabilități ce implică pe $X$ și $Y$.
		% AICI DE MENTIONAT SI SLOTUL "VAL" AL LUI DRAGOS!
	\end{itemize}\vspace*{1\baselineskip} \par
	Motivul pentru care este necesară specificarea suportului densității la crearea unui obiect de tip \textbf{contRV} este următorul: la calculul integralelor unde unul dintre capete nu este un număr finit, comportamentul funcției \lstinline|integrate()| poate produce rezultate neașteptate(de introdus exemple). Astfel, restricționând domeniul la punctele în care integrandul ia valori nenule, calculul integralei devine mult mai precis.
	În stadiul actual însă, permitem ca suportul densitații să fie specificat doar ca o reuniune de intervale inchise în cazul v.a unidimensionale, sau dreptunghiuri în cazul celor bidimensionale.\vspace*{1\baselineskip}\par
	
	Definiția clasei este următoarea:
	\begin{lstlisting}
		setClass("contRV", representation (
		densitate="function",
		val="function",
		bidimen="logical",
		suport="list",
		ref_va_bidimen = "contRV_or_NULL"
		))
		
		# Am folosit acest union pentru a permite referintei catre v.a bidimen sa fie nula
		setClassUnion("contRV_or_NULL", c("contRV", "NULL"))
	\end{lstlisting}\pagebreak
	
	Pentru a crea un obiect \textbf{contRV}, am definit următorul constructor:
	\begin{lstlisting}
		contRV <- function(densitate, val = function(x) x, bidimen = FALSE, suport = list(c(-Inf, Inf)), ref_va_bidimen = NULL)
		{
			if(length(suport) < 2)
			suport <- list(suport, list())
			if (bidimen & missing(val))
			val = function(x, y) x * y
			
			obj <- new("contRV", densitate = densitate, val = val, bidimen = bidimen,
			suport = suport, ref_va_bidimen = ref_va_bidimen)
			
			return (obj)
		}
		
	\end{lstlisting}

	De exemplu, pentru crearea unui obiect \textbf{contRV} ce reprezintă o variabilă aleatoare continuă bidimensională $(X, Y)$ cu densitatea:
	\[ 
	f(x, y)= \left\{
	\begin{array}{ll}
		\frac{6}{7}(x+y)^2, & x \in [0, 1] \times [0, 1] \\\\
		0,				   & \text{în rest}
	\end{array} 
	\right. 
	\]
	Scriem:
	\begin{lstlisting}[numbers=none]
		XY <- contRV(densitate = function (x, y) 6/7(x+y)^2, bidimen = TRUE, suport = list(list(c(0, 1)), list(c(0, 1))))
	\end{lstlisting}\vspace*{1\baselineskip}
	
	
	De asemenea, clasa \textbf{contRV} pune la dispoziție utilizatorilor pachetului o colecție de metode pentru a efectua operații pe variabile aleatoare continue, precum: calculul probabilităților(cu ajutorul metodei $P$, detalii în exercițiul 7), calculul mediilor și a dispersiilor(metodele $E$ și $Var$, detalii în exercițiile 5-6) și obținerea densitaților marginale(metodele $marginalaX$ și $marginalaY$).
\end{document}