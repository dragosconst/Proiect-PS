\documentclass[12pt]{article}
\usepackage{fancyhdr}
\usepackage{amsmath}
\usepackage{amsthm}
\usepackage{mathtools}
\usepackage{enumitem}
\usepackage[Export]{adjustbox}
\usepackage{cancel}
\usepackage{algorithm}
\usepackage[noend]{algpseudocode}
\usepackage{graphicx}
\usepackage[margin = 1in]{geometry}
\usepackage{blindtext}
\usepackage[section]{placeins}
\usepackage{xcolor,listings}
\usepackage{textcomp}
\usepackage[utf8]{inputenc}
\lstset{upquote=true}
\lstdefinestyle{myCustomRStyle}{
	language=R,
	backgroundcolor = \color{lightgray!10!white},
	numbers=left,
	stepnumber=1,
	numbersep=10pt,
	tabsize=2,
	showspaces=false,
	breaklines=true
	showstringspaces=false,
	basicstyle=\footnotesize\ttfamily,
	keywordstyle=\bfseries\color{blue!50!black},
	commentstyle=\itshape\color{orange!60!black},
	identifierstyle=\color{black},
	stringstyle=\color{green!50!black},
	xleftmargin=\parindent,
	frame=L
}
\lstset{literate=%
	*{0}{{{\color{blue}0}}}1
	{1}{{{\color{blue}1}}}1
	{2}{{{\color{blue}2}}}1
	{3}{{{\color{blue}3}}}1
	{4}{{{\color{blue}4}}}1
	{5}{{{\color{blue}5}}}1
	{6}{{{\color{blue}6}}}1
	{7}{{{\color{blue}7}}}1
	{8}{{{\color{blue}8}}}1
	{9}{{{\color{blue}9}}}1
}

\lstset{basicstyle=\small,style=myCustomRStyle}
\usepackage{amsfonts}
\graphicspath{ {./images/} }

\pagestyle{fancy}
\fancyhead{}
\fancyfoot{}
\usepackage[T1]{fontenc}
\fancyhead[L]{Placeholder}
\fancyhead[R]{Placeholder}
\setlength{\headheight}{25pt}
\fancyfoot[C]{\thepage}
\title{Placeholder}
\author{Placeholder}

\begin{document}
	%\section*{Cerința 7}
	\textbf{Crearea unei funcții P care permite calculul diferitelor tipuri de probabilități asociate
		unei variabile aleatoare continue(similar funcției P din pachetul discreteRV) }\vspace{5mm}
	
	Pentru a calcula probabilități pe variabile aleatoare continue, am definit funcția $P$ ca o metodă a clasei \textbf{contRV}, astfel:
	\begin{lstlisting}[numbers=none]
		setMethod("P", "contRV",
		function (object) {
			return (integrala(object)) # integreaza pe suport
		})
	\end{lstlisting}

	După cum se poate observa, parametrul funcției este un obiect de tip \textbf{contRV}. Acest lucru poate părea ciudat la prima vedere; ar fi inutil să putem calcula doar probabilități de tipul $P(X)$, intrucât rezultatul ar fi întodeauna egal cu 1. În contextul pachetului, însă, un obiect de tip \textbf{contRV} nu reprezintă întotdeauna o variabilă aleatoare propriu-zisă. Mai precis, orice obiect poate fi rezultatul unei expresii de tipul: $X \leq x$, $(X \leq x) \cap  (Y \geq y)$, $(X < a) \cup (X > b)$ etc. Astfel, prin evaluarea expresiilor, restrângem domeniul pe care integrăm densitățile în calculul probabilităților și să păstrăm notații cât mai apropiate de cele matematice(detalii în exemplele de la sfârșit).\par
	
	Pentru a evalua expresiile, am supraîncărcat următorii operatori:
	\begin{lstlisting}
		setMethod("<", c("contRV", "numeric"), function (e1, e2) {
			comp(e1, e2, "<=") # P(X < x) = P(X <= x)
		})
		setMethod("<=", c("contRV", "numeric"), function (e1, e2) {
			comp(e1, e2, "<=")
		})
		setMethod(">", c("contRV", "numeric"), function (e1, e2) {
			comp(e1, e2, ">=") # P(X > x) = P(X >= x)
		})
		setMethod(">=", c("contRV", "numeric"), function (e1, e2) {
			comp(e1, e2, ">=")
		})
		setMethod("==", c("contRV", "numeric"), function (e1, e2) {
			comp(e1, e2, "==")
		})
		setMethod("%AND%", c("contRV", "contRV"), function (e1, e2) {
		op(e1, e2, "&") # intersectie
		})
		setMethod("%OR%", c("contRV", "contRV"), function (e1, e2) {
		op(e1, e2, "|") # reuniune
		})
		setMethod("|", c("contRV", "contRV"), function (e1, e2) {
		cond(e1, e2) # conditionare
		})
	\end{lstlisting}\pagebreak

	Pentru operatorii de inegalitate și egalitate, se observă ca aceștia au drept parametri un obiect \textbf{contRV} și un număr real. Se apelează funcția $comp$, ce va efectua restrângerea efectivă a suportului variabilei aleatoare pentru calculul probabilităților.
	
	\begin{lstlisting}
		# Determina suportul pentru expresii de tipul X <= x, X >= x
		comp <- function(X, x, c)
		{
			
			if (X@bidimen)
				stop("Nu se poate compara o v.a bidimensionala cu un numar real!")
			
			suportNou <- list()
			nr <- 1
			
			# Presupunem ca intervalele sunt in ordine crescatoare dupa capatul inferior
			# si nu se intersecteaza!	
			if (c == "==")
			{
				for (i in X@suport[[1]])
				{
					a <- i[1]
					b <- i[2]
					
					if (x < a)
					break # nu mai are rost sa cautam
					
					if (a <= x & b >= x) # daca x se afla in intervalul [a, b]
					{
						suportNou[[nr]] <- c(x, x) # suportul va fi intervalul [x, x]
						break
					}
				}
			}
			else if (c == "<=")
			{
				# Exemplu: Daca suportul densitatii este format din [0, 2] U [4, 7] U [9, 11]
				# Noul suport pentru X <= 5 va fi [0, 2] U [4, 5]
				
				for (i in X@suport[[1]])
				{
					a <- i[1]
					b <- i[2]
					
					if (x < a)  # daca x este mai mic decat capatul inferior al intervalului
					break # am terminat de construit suportul
					
					if (a <= x & b >= x) # daca x se afla in intervalul [a, b]
					{
						suportNou[[nr]] <- c(a, x) # adaugam ultimul interval din noul suport, adica [a, x]
						break
					}
					
					# altfel, n-am ajuns la un interval care sa-l contina pe x, deci il adaugam in suportul nou
					suportNou[[nr]] <- c(a, b)
					nr <- nr + 1
				}
			}
			else # ">="
			{
				# Exemplu: Daca suportul densitatii este format din [0, 2] U [4, 7] U [9, 11]
				# Noul suport pentru X >= 5 va fi [5, 7] U [9, 11]
				
				# parcurgem intervalele in ordine descrescatoare dupa capatul inferior
				for (i in rev(X@suport[[1]]))
				{
					a <- i[1]
					b <- i[2]
					
					if (x > b)
					break
					
					if (a <= x & b >= x) # daca x se afla in intervalul [a, b]
					{
						suportNou[[nr]] <- c(x, b) # adaugam ultimul interval din noul suport, adica [x, b]
						break
					}
					
					# altfel, inca n-am ajuns la un interval care sa-l contina pe x, deci il adaugam in suportul nou
					suportNou[[nr]] <- c(a, b)
					nr <- nr + 1
				}
				
				# inversam ordinea din noul suport, intrucat am parcurs intervalele din suport in ordine inversa
				suportNou <- rev(suportNou)
			}
			
			# atentie! rezultatul obtinut nu mai este o v.a! se foloseste doar pt a calcula probabilitati!
			return (contRV(densitate = X@densitate, val = X@val, bidimen = X@bidimen, suport = suportNou,
			ref_va_bidimen = X@ref_va_bidimen))
		}
	\end{lstlisting}\pagebreak
	
	Pentru operatorii \%AND\% și \%OR\% ce implementează intersecția, respectiv reuniunea de variabile aleatoare continue, se apelează funcția $op$, care va avea un comportament diferit în funcție de relația dintre cei doi parametri $X$ și $Y$ de tip \textbf{contRV}. Mai exact, distingem următoarele cazuri:
	\begin{itemize}
		\item $X$ și $Y$ au aceeași densitate de probabilitate(apare de obicei în urma unor expresii de genul: $(X < a) \cap (X > b)$ ), caz în care formăm un obiect de tip \textbf{contRV} cu densitatea cunoscută și drept suport intersecția dintre suporturile lui $X$ și $Y$.
		\item $X$ și $Y$ au o referință către aceeași v.a bidimensională, așa că formăm un nou obiect nou obiect \textbf{contRV} cu densitatea comună deja cunoscută și drept suport produsul cartezian intre suportul lui $X$ și cel al lui $Y$
		\item $X$ și $Y$ au câte o referință către v.a bidimensionale diferite(sau nu au nicio referință), caz în care le considerăm independente și formăm un nou obiect \textbf{contRV} cu densitatea comună $f(x, y) = f_X(x) * f_Y(y)$ și drept suport produsul cartezian intre suportul lui $X$ și cel al lui $Y$		
	\end{itemize}

	Funcția $op$ este definită astfel:
	\begin{lstlisting}
		# Reuniune si intersectie de variabile aleatoare
		op <- function(X, Y, o)
		{	
			suportNou <- list()
			nr <- 1
			
			if (o == "&") # intersectie
			{
				
				if (!identical(X@densitate, Y@densitate))
				{
					
					XY <- NULL
					
					if (!is.null(X@ref_va_bidimen) & identical(X@ref_va_bidimen, Y@ref_va_bidimen))
					{
						# aici se face o copie
						XY <- X@ref_va_bidimen
					}
					else
					{
						# consideram ca sunt independente
						XY <- contRV(densitate = function(x, y) {X@densitate(x) * Y@densitate(y)}, bidimen = TRUE)
					}
					
					XY@suport[[1]] <- X@suport[[1]]
					XY@suport[[2]] <- Y@suport[[1]]
					
					return (XY)
				}
				
				for (i in X@suport[[1]])
				{
					for (j in Y@suport[[1]])
					{
						# reuniuni de intersectii ale intervalelor din suport
						
						A <- interval_intersect(i, j)
						if (!is.null(A))
						{
							suportNou[[nr]] <- A
							nr <- nr + 1
						}
					}
				}
				
				return (contRV(densitate = X@densitate, val = X@val, bidimen = X@bidimen, suport = suportNou,
				ref_va_bidimen = X@ref_va_bidimen)) # intoarce contRV pt a integra suportul ramas
			}
			else # reuniune
			{
				return (P(X) + P(Y) - P(X %AND% Y)) # aici intoarce deja probabilitatea calculata
				# problema este ca nu se mai pot aplica alte operatii pe v.a
			}
		}
	\end{lstlisting}\par
	O limitare a acestei funcții ar fi că în urma unei operații de reuniune, nu se întoarce un obiect \textbf{contRV} pentru a fi folosit în alte expresii, ci direct rezultatul probabilitații, adică $P(X \in A) + P(Y \in B) - P(X \in A, Y \in B)$.\vspace*{2\baselineskip}\par
	
	Ultimul operator implementat este cel de condiționare:
	\begin{lstlisting}[numbers=none]
		setMethod("|", c("contRV", "contRV"), function (e1, e2) {
			cond(e1, e2)
		})
	\end{lstlisting}
	
	pentru care se apelează funcția $cond$, care are la rândul ei un comportament diferit în funcție de relația dintre cele 2 obiecte $X$ și $Y$ de tipul \textbf{contRV}. Dacă:
	\begin{itemize}
		\item $X$ și $Y$ au densitățile marginale dintr-o v.a bidimensională $(X, Y)$, atunci calculează direct probabilitatea folosind formula: $P(X \in A \mid Y = y) = \int_{A} f_{X|Y}(x|y) \,dx$, unde densitatea condiționată o obținem din funcția \texttt{dens\_condit\_x\_de\_y}, descrisă în exercițiul 11.
		\item Altfel, avem ori $X$ și $Y$ independente, ori au aceeași densitate. În ambele cazuri, putem folosi formula: $P(X \in A \mid Y \in B) = \frac{P(X \in A, Y \in B)}{P(Y \in B)}$. Chiar și pentru $X$, $Y$ independente, rezultatul ar fi cel așteptat, adică $P(X \in A)$.
	\end{itemize}\pagebreak

	Funcția \texttt{cond} arată în felul următor:
	\begin{lstlisting}
		# Calculeaza probabilitatea conditionata
		# X si Y pot fi expresii de tipul Z <= x, Z %AND% W etc.
		cond <- function (X, Y)
		{
			if (!is.null(X@ref_va_bidimen) & identical(X@ref_va_bidimen, Y@ref_va_bidimen))
			{
				# aici probabil ar trebui verificat daca X si Y referintiaza aceeasi v.a bidimensionala
				XY <- X@ref_va_bidimen
				fx_cond_y <- dens_condit_x_de_y(XY)
				
				if (length(Y@suport[[1]]) == 0) # inseamna ca nu a fost gasit y in suportul lui Y
				{
					return (0)
				}
				
				if (length(Y@suport[[1]]) != 1  || Y@suport[[1]][[1]][[1]] != Y@suport[[1]][[1]][[2]])
				{
					stop("Nu pot calcula asa ceva! Suportul lui Y trebuie sa fie un singur punct!!")
				}
				
				
				yfixat <- Y@suport[[1]][[1]][[1]]
				
				sum <- 0
				for (i in X@suport[[1]]) {
					tryCatch(sum <- sum + integrate(fx_cond_y, i[1], i[2], y = yfixat, abs.tol = 1.0e-13)$value,
					error= function(err)
					{
						stop("Integrala a esuat.")
					})
				}
				return (sum)
				
			}
			else
			{
				return (integrala(X %AND% Y) / integrala(Y)) # P(X intersect Y) / P(Y)
			}
		}
	\end{lstlisting}\pagebreak

	Comparație între notații:
	\begin{center}
		\begin{tabular}{|| c | c ||}
			\hline
			\textcolor{blue}{contRV} & Probabilități \\
			
			\hline
			\lstinline|P(X <= x)| & $P(X \leq x)$ \\
			
			\hline
			\lstinline|P((X <= a)  %AND% (Y >= b))| & $P(X \leq a, Y \geq b)$ \\
			
			\hline
			\lstinline|P((X <= a)  %OR% (X >= b))| & $P(X \leq a \cup X \geq b)$ \\
			
			
			\hline
			\lstinline!P((X >= a)  %AND% (X <= b)  | X < c) ! & $P(a \leq X \leq b \mid X < c)$ \\
			
			\hline
			\lstinline!P(X > x | Y == y)! & $P(X > x \mid Y = y)$ \\
			
			\hline
			
		\end{tabular}
	\end{center}\vspace*{3\baselineskip}

	Exemple de cod:
	\begin{lstlisting}[numbers=none]
		> XY <- contRV(densitate = function (x, y) (6/7) * (x+y)^2,
		             bidimen = TRUE,
		             suport = list(list(c(0, 1)), list(c(0, 1))))
		> X <- marginalaX(XY)
		> Y <- marginalaY(XY)
		> P((X <= 0.5) %AND% (X >= 0.2) | Y == 0.2)
		[1] 0.1622093
		> P((X <= 0.7) %AND% (Y >= 0.5))
		[1] 0.3815
		
		
		> func <- function(x)
		+ {
			     if (x < -1)
			         0
			     else if (x < 0)
			         1 + x
			     else if (x < 1)
			         1 - x
			     else
			         0
			}
		> Z <- contRV(densitate = Vectorize(func), bidimen = FALSE, suport = list(c(-1, 1)))
		> P(Z <= 0)
		[1] 0.5
		> P(((Z <= 0.5) %AND% (Z >= -0.7)) %OR% (Z <= 1))
		[1] 1
	\end{lstlisting}
\end{document}