\documentclass[12pt]{article}
\usepackage{fancyhdr}
\usepackage{amsmath}
\usepackage{amsthm}
\usepackage{mathtools}
\usepackage{enumitem}
\usepackage[Export]{adjustbox}
\usepackage{cancel}
\usepackage{algorithm}
\usepackage[noend]{algpseudocode}
\usepackage{graphicx}
\usepackage[margin = 1in]{geometry}
\usepackage{blindtext}
\usepackage[section]{placeins}
\usepackage{xcolor,listings}
\usepackage{textcomp}
\usepackage[utf8]{inputenc}
\lstset{upquote=true}
\lstdefinestyle{myCustomRStyle}{
	language=R,
	backgroundcolor = \color{lightgray!10!white},
	numbers=left,
	stepnumber=1,
	numbersep=10pt,
	tabsize=2,
	showspaces=false,
	breaklines=true
	showstringspaces=false,
	basicstyle=\footnotesize\ttfamily,
	keywordstyle=\bfseries\color{blue!50!black},
	commentstyle=\itshape\color{orange!60!black},
	identifierstyle=\color{black},
	stringstyle=\color{green!50!black},
	xleftmargin=\parindent,
	frame=L
}
\lstset{literate=%
	*{0}{{{\color{blue}0}}}1
	{1}{{{\color{blue}1}}}1
	{2}{{{\color{blue}2}}}1
	{3}{{{\color{blue}3}}}1
	{4}{{{\color{blue}4}}}1
	{5}{{{\color{blue}5}}}1
	{6}{{{\color{blue}6}}}1
	{7}{{{\color{blue}7}}}1
	{8}{{{\color{blue}8}}}1
	{9}{{{\color{blue}9}}}1
}

\lstset{basicstyle=\small,style=myCustomRStyle}
\usepackage{amsfonts}
\graphicspath{ {./images/} }

\pagestyle{fancy}
\fancyhead{}
\fancyfoot{}
\usepackage[T1]{fontenc}
\fancyhead[L]{Placeholder}
\fancyhead[R]{Placeholder}
\setlength{\headheight}{25pt}
\fancyfoot[C]{\thepage}
\title{Placeholder}
\author{Placeholder}

\begin{document}
	
	\textbf{8.) Afișarea unei “fișe de sinteză” care să conțină informații de bază despre respectiva
		repartiție(cu precizarea sursei informației!). Relevant aici ar fi să precizați pentru ce e
		folosită in mod uzual acea repartiție, semnificația parametrilor, media, dispersia etc.}\vspace{5mm}\\
\indent Scurtă descriere a funcției: \\
\begin{center}
	\begin{tabular}{|| c | c | c ||}
		\hline
		Parametrul & Tipul & Descriere \\
		\hline
		\textcolor{blue}{Rep}	 & \textcolor{violet}{Unul din vectorii definiți în pachet} & Repartiția dată ca parametru \\
		\hline
	\end{tabular}
\end{center}	
	\begin{lstlisting}
		#Functia afiseaza informatiile despre o repartitie mai compact
		#compact = fara spatii goale
		#De asemenea reprezinta o metoda mai intuitiva de a afisa
		Fisa_sinteza <- function(Rep){
			for (i in Rep)
			{
				print(i)
			}
			
		}
	\end{lstlisting}
	
	
	Funcția Fisa\underline{\hspace{.08in}}sinteza afișează informațiile salvate despre repartiția dată ca parametru sub forma unei fișe de sinteză.\hfill \break
	\indent Informațiile sunt salvate într-o listă de elemente de tip String.
	
	
\end{document}