\documentclass[12pt, twoside]{article}
\usepackage[utf8]{inputenc}
\usepackage{listings}
\lstdefinestyle{myCustomRStyle}{
	language=R,
	numbers=left,
	stepnumber=1,
	numbersep=10pt,
	tabsize=2,
	showspaces=false,
	showstringspaces=false
}
\lstset{basicstyle=\small,style=myCustomRStyle}
\usepackage{amsfonts}

\title{ex 8}
\author{Ristea Mihai Cristian}
\date{February 2021}

\begin{document}
	
	\textbf{8) Afișarea unei “fișe de sinteză” care să conțină informații de bază despre respectiva
		repartiție(cu precizarea sursei informației!). Relevant aici ar fi să precizați pentru ce e
		folosită in mod uzual acea repartiție, semnificația parametrilor, media, dispersia etc.}\vspace{5mm}
	
	\begin{lstlisting}
		#Functia afiseaza informatiile despre o repartitie mai compact
		#compact = fara spatii goale
		#De asemenea reprezinta o metoda mai intuitiva de a afisa
		Fisa_sinteza <- function(Rep){
			for (i in Rep)
			{
				print(i)
			}
			
		}
	\end{lstlisting}
	
	\begin{center}
		\begin{tabular}{ c c }
			Fisa\underline{\hspace{.08in}}sinteza & Denumirea functiei \\
			Rep & Repartitia data ca parametru \\
			for (i in rep) & Parcurgerea informatiilor din repartitie 
		\end{tabular}
	\end{center}
	\vspace{30mm}
	
	Functia Fisa\underline{\hspace{.08in}}sinteza afiseaza informatiile salvate despre repartitia data ca parametru sub forma unei fise de sinteza.\hfill \break
	\indent Informmatiile sunt salvate intr-o lista de elemente de tip String. Parcurgand lista afisam elementele acesteia pe rand. Aceasta abordare permite afisarea informatiilor compact, apelarea listei in consola afiseaza unele randuri goale in anumite cazuri.
	
	
\end{document}